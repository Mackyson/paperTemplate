\chapter{従来手法}
\label{chap:conv}

%----------------------------------------------
\section{まえがき}
%----------------------------------------------
本章では,何らか法の従来手法を説明する.
まず\ref{sec:conv:something}節では,何らかの分野で従来より用いられる何とかについて何とかを導入する.
\ref{sec:conv:somewhat}節では,なんとかについて述べる.

%----------------------------------------------
\section{何らか法}
\label{sec:conv:something}
%----------------------------------------------

何らか法とは,何らか何らか何とかかんとかであり,次式で表される
\begin{align}
  \hat{\bm{\theta}} = \argmin_{\bm{\theta} \in S} \ \mathcal{J}(\bm{\theta})
\end{align}

%----------------------------------------------
\section{何とか法}
\label{sec:conv:somewhat}
%----------------------------------------------

%----------------------------------------------
\subsection{表記の定義}
\label{sec:conv:somewhat:definition}
%----------------------------------------------
何とかかんとかと書ける.

%----------------------------------------------
\subsection{何とか法の導出}
\label{sec:conv:somewhat:derivation}
%----------------------------------------------
何とかかんとかと導ける.

%----------------------------------------------
\section{本章のまとめ}
%----------------------------------------------
本章では,何らか法の従来手法について説明した.
次章以降では,より詳細な何とかや何らか法の適用範囲の拡大を達成するために,
\ref{sec:conv:somewhat:derivation}節で導入した何とか法の発展的な理論拡張を提案する.
